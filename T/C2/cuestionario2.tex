\documentclass[11pt,a4paper]{article}
\usepackage[spanish,es-nodecimaldot]{babel}	% Utilizar español
\usepackage[utf8]{inputenc}					% Caracteres UTF-8
\usepackage{graphicx}						% Imagenes
\usepackage[hidelinks]{hyperref}			% Poner enlaces sin marcarlos en rojo
\usepackage{fancyhdr}						% Modificar encabezados y pies de pagina
\usepackage{float}							% Insertar figuras
\usepackage[textwidth=390pt]{geometry}		% Anchura de la pagina
\usepackage[nottoc]{tocbibind}				% Referencias (no incluir num pagina indice en Indice)
\usepackage{enumitem}						% Permitir enumerate con distintos simbolos
\usepackage[T1]{fontenc}					% Usar textsc en sections
\usepackage{amsmath}						% Símbolos matemáticos
\usepackage{bbold}

% Comando para poner el nombre de la asignatura
\newcommand{\asignatura}{Aprendizaje Automático}

% Comandos utilies
\newcommand{\answer}{\noindent\textbf{Solución}}
\newcommand{\ein}{E$_{in}$}
\newcommand{\eout}{E$_{out}$}
\newcommand{\addtoc}[1]{\addcontentsline{toc}{section}{#1}}

% Configuracion de encabezados y pies de pagina
\pagestyle{fancy}
\lhead{Vladislav Nikolov Vasilev}
\rhead{\asignatura{}}
\lfoot{Grado en Ingeniería Informática}
\cfoot{}
\rfoot{\thepage}
\renewcommand{\headrulewidth}{0.4pt}		% Linea cabeza de pagina
\renewcommand{\footrulewidth}{0.4pt}		% Linea pie de pagina

\begin{document}
\pagenumbering{gobble}

% Pagina de titulo
\begin{titlepage}

\begin{minipage}{\textwidth}

\centering

\includegraphics[scale=0.5]{img/ugr.png}\\

\textsc{\Large \asignatura{}\\[0.2cm]}
\textsc{GRADO EN INGENIERÍA INFORMÁTICA}\\[1cm]

\noindent\rule[-1ex]{\textwidth}{1pt}\\[1.5ex]
\textsc{{\Huge Trabajo 2\\[0.5ex]}}
\textsc{{\Large Cuestiones de Teoría\\}}
\noindent\rule[-1ex]{\textwidth}{2pt}\\[3.5ex]

\end{minipage}

\vspace{0.5cm}

\begin{minipage}{\textwidth}

\centering

\textbf{Autor}\\ {Vladislav Nikolov Vasilev}\\[2.5ex]
\textbf{Rama}\\ {Computación y Sistemas Inteligentes}\\[2.5ex]
\vspace{0.3cm}

\includegraphics[scale=0.3]{img/etsiit.jpeg}

\vspace{0.7cm}
\textsc{Escuela Técnica Superior de Ingenierías Informática y de Telecomunicación}\\
\vspace{1cm}
\textsc{Curso 2018-2019}
\end{minipage}
\end{titlepage}

\pagenumbering{arabic}
\tableofcontents
\thispagestyle{empty}				% No usar estilo en la pagina de indice

\newpage

\setlength{\parskip}{1em}

\section*{Ejercicio 1}
\addtoc{Ejercicio 1}

\noindent Identificar de forma precisa dos condiciones imprescindibles para que un problema de predicción puede ser aproximado
por inducción desde una muestra de datos. Justificar la respuesta usando los resultados teóricos estudiados.

\answer

\section*{Ejercicio 2}
\addtoc{Ejercicio 2}

\noindent El jefe de investigación de una empresa con mucha experiencia en problemas de predicción de datos tras analizar los
resultados de los muchos algoritmos de aprendizaje usados sobre todos los problemas en los que la empresa ha trabajado a lo
largo de su muy dilatada existencia, decide que para facilitar el mantenimiento del código de la empresa van a seleccionar un
único algoritmo y una única clase de funciones con la que aproximar todas las soluciones a sus problemas presentes y futuros.
¿Considera que dicha decisión es correcta y beneficiará a la empresa? Argumentar la respuesta usando los resultados teóricos
estudiados.

\answer

\section*{Ejercicio 3}
\addtoc{Ejercicio 3}

\noindent ¿Que se entiende por una solución PAC a un problema de aprendizaje? Identificar el porqué de la incertidumbre e
imprecisión.

\answer

En el ámbito del aprendizaje, una solución PAC significa que es \textit{Probably Approximately Correct}, 
lo cuál traducido al español vendría a ser algo así como ``correcta probablemente aproximada''.
Veamos qué significa todo esto sobre la desigualdad de Hoeffding aplicada al problema de aprendizaje:

\begin{equation}
\label{eq:hoeffding}
	\mathbb{P}(\mathcal{D}: | \; \text{E}_{in} - \text{E}_{out} | \; > \varepsilon) \leq 2e^{-2\varepsilon^2N}
\end{equation}

\begin{itemize}[label=\textbullet]
	\item  La parte de ``probablemente'' hace referencia a una alta probabilidad. En la expresión mostrada en
	\eqref{eq:hoeffding}, se puede ver una probabilidad de que algo malo suceda, es decir, que la
	diferencia 	entre los valores de \ein{} y \eout{} sea mayor que un $\varepsilon$ dado, o lo que es lo mismo, que los
	errores disten mucho entre sí. Como en la 	expresión de la parte derecha nos encontramos con un exponencial negativo,
	con los valores adecuados de $\varepsilon$ y $N$ podemos hacer que esa probabilidad de que algo malo pase sea pequeña.
	Por tanto, la probabilidad de que la diferencia sea menor que $\varepsilon$ vendría dada por:
	
	\[\mathbb{P}(\mathcal{D}: | \; \text{E}_{in} - \text{E}_{out} | \; < \varepsilon) \geq 1 - 2e^{-2\varepsilon^2N}\]
	
	\noindent la cuál sí que tendría un valor muy alto, siendo por tanto más ``probable'' que esa diferencia sea menor que
	$\varepsilon$.
	\item La parte de ``aproximada'' indica que \ein{} no es exactamente igual que \eout{}, pero que ambos valores están muy
	próximos. Esta aproximación viene dada por el valor de $\varepsilon$.
\end{itemize}

La \textbf{incertidumbre} viene dada por la probabilidad. Nunca se puede tener la certeza de que el resultado sea 100\%
correcto, pero se puede afirmar con una alta probabilidad de que así sea (por eso es PAC).
La \textbf{imprecisión}, por otro lado, viene dada por el valor de $\varepsilon$. Es decir, los valores de \ein{} y \eout{},
al estar aprendiendo de una muestra la cuál puede tener un tamaño no lo suficientemente grande o no ser muy representativa
de la población, van a ser diferentes.
Si pudiésemos aprender de toda la población directamente, en ese caso $\varepsilon$ sería 0, ya que los dos errores serían
iguales, pero habría que pagar muchos costes de tiempo, potencia de cómputo y almacenamiento. Por tanto, al estar siempre
aprendiendo de una muestra y no de la población entera nos vamos a encontrar con estos dos problemas.

\section*{Ejercicio 7}
\addtoc{Ejercicio 7}

\noindent ¿Por qué la desigualdad de Hoeffding definida para clases $\mathcal{H}$ de una única función no es aplicable de
forma directa cuando el número de hipótesis de $\mathcal{H}$ es mayor de 1? Justificar la respuesta.

\answer



\newpage

\begin{thebibliography}{5}

\bibitem{nombre-referencia}
Texto referencia
\\\url{https://url.referencia.com}

\end{thebibliography}

\end{document}

