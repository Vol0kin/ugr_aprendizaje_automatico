\documentclass[11pt,a4paper]{article}
\usepackage[spanish,es-nodecimaldot]{babel}	% Utilizar español
\usepackage[utf8]{inputenc}					% Caracteres UTF-8
\usepackage{graphicx}						% Imagenes
\usepackage[hidelinks]{hyperref}			% Poner enlaces sin marcarlos en rojo
\usepackage{fancyhdr}						% Modificar encabezados y pies de pagina
\usepackage{float}							% Insertar figuras
\usepackage[textwidth=390pt]{geometry}		% Anchura de la pagina
\usepackage[nottoc]{tocbibind}				% Referencias (no incluir num pagina indice en Indice)
\usepackage{enumitem}						% Permitir enumerate con distintos simbolos
\usepackage[T1]{fontenc}					% Usar textsc en sections
\usepackage{amsmath}						% Símbolos matemáticos
\usepackage{algpseudocode}
\usepackage{algorithm}

% no accents in math operators
\unaccentedoperators

% Comando para poner el nombre de la asignatura
\newcommand{\asignatura}{Aprendizaje Automático}
\newcommand{\autor}{Vladislav Nikolov Vasilev}

% Comandos utilies
\newcommand{\answer}{\noindent\textbf{Solución}}
\newcommand{\ein}{E$_{in}$}
\newcommand{\eout}{E$_{out}$}
\newcommand{\addtoc}[1]{\addcontentsline{toc}{section}{#1}}

% Configuracion de encabezados y pies de pagina
\pagestyle{fancy}
\lhead{\autor{}}
\rhead{\asignatura{}}
\lfoot{Grado en Ingeniería Informática}
\cfoot{}
\rfoot{\thepage}
\renewcommand{\headrulewidth}{0.4pt}		% Linea cabeza de pagina
\renewcommand{\footrulewidth}{0.4pt}		% Linea pie de pagina

\begin{document}
\pagenumbering{gobble}

% Pagina de titulo
\begin{titlepage}

\begin{minipage}{\textwidth}

\centering

\includegraphics[scale=0.5]{img/ugr.png}\\

\textsc{\Large \asignatura{}\\[0.2cm]}
\textsc{GRADO EN INGENIERÍA INFORMÁTICA}\\[1cm]

\noindent\rule[-1ex]{\textwidth}{1pt}\\[1.5ex]
\textsc{{\Huge TRABAJO 3\\[0.5ex]}}
\textsc{{\Large Cuestiones de Teoría\\}}
\noindent\rule[-1ex]{\textwidth}{2pt}\\[3.5ex]

\end{minipage}

\vspace{0.5cm}

\begin{minipage}{\textwidth}

\centering

\textbf{Autor}\\ {\autor{}}\\[2.5ex]
\textbf{Rama}\\ {Computación y Sistemas Inteligentes}\\[2.5ex]
\vspace{0.3cm}

\includegraphics[scale=0.3]{img/etsiit.jpeg}

\vspace{0.7cm}
\textsc{Escuela Técnica Superior de Ingenierías Informática y de Telecomunicación}\\
\vspace{1cm}
\textsc{Curso 2018-2019}
\end{minipage}
\end{titlepage}

\pagenumbering{arabic}
\tableofcontents
\thispagestyle{empty}				% No usar estilo en la pagina de indice

\newpage

\setlength{\parskip}{1em}

\section*{Ejercicio 1}
\addtoc{Ejercicio 1}

\noindent ¿Podría considerarse Bagging como una técnica para estimar el error de predicción de un
modelo de aprendizaje? Diga si o no con argumentos. En caso afirmativo compárela con
validación cruzada.

\answer

\section*{Ejercicio 2}
\addtoc{Ejercicio 2}

\noindent Considere que dispone de un conjunto de datos linealmente separable. Recuerde que una
vez establecido un orden sobre los datos, el algoritmo perceptron encuentra un hiperplano
separador interando sobre los datos y adaptando los pesos de acuerdo al algoritmo

\begin{algorithm}[H]
\caption{Perceptron}
\begin{algorithmic}[1]
\State \textbf{Entradas}: $(\mathbf{x}_i, y_i) = 1, \dots, n \; , \; w=0, \; k = 0$
\Repeat
	\State $k \gets (k + 1) \; \mod \; n$
	\If{$\text{sign}(y_i) \neq \text{sign}(\mathbf{w}^T\mathbf{x}_i)$}
		\State $\mathbf{w} \gets \mathbf{w} + y_i\mathbf{x}_i$
	\EndIf
\Until{todos los puntos bien clasificados}
\end{algorithmic}
\end{algorithm}

\noindent Modificar este pseudo-código para adaptarlo a un algoritmo simple de SVM, considerando
que en cada iteración adaptamos los pesos de acuerdo al caso peor clasificado de toda la
muestra. Justificar adecuadamente/matematicamente el resultado, mostrando que al final
del entrenamiento solo estaremos adaptando los vectores soporte.

\answer

\section*{Ejercicio 3}
\addtoc{Ejercicio 3}

\noindent Considerar un modelo SVM y los siguientes datos de entrenamiento: Clase-1:$\lbrace (1,1),$
$(2,2),(2,0) \rbrace$, Clase-2:$\lbrace (0,0),(1,0),(0,1) \rbrace$

\begin{enumerate}[label=\textit{\alph*})]
	\item Dibujar los puntos y construir por inspección el vector de pesos para el hiperplano óptimo y el margen óptimo.
\end{enumerate}

\answer

\begin{enumerate}[resume,label=\textit{\alph*})]
	\item ¿Cuáles son los vectores soporte?
\end{enumerate}

\answer

\begin{enumerate}[resume,label=\textit{\alph*})]
	\item Construir la solución en el espacio dual. Comparar la solución con la del apartado (a)
\end{enumerate}

\answer



\section*{Ejercicio 4}
\addtoc{Ejercicio 4}

\noindent ¿Cúal es el criterio de optimalidad en la construcción de un árbol? Analice un clasificador
en árbol en términos de sesgo y varianza. ¿Que estrategia de mejora propondría?

\answer

\section*{Ejercicio 5}
\addtoc{Ejercicio 5}

\noindent ¿Cómo influye la dimensión del vector de entrada en los modelos: SVM, RF, Boosting and
NN?

\answer

\section*{Ejercicio 6}
\addtoc{Ejercicio 6}

\noindent El método de Boosting representa una forma alternativa en la búsqueda del mejor clasificador
respecto del enfoque tradicional implementado por los algoritmos PLA, SVM, NN, etc. a)
Identifique de forma clara y concisa las novedades del enfoque; b) Diga las razones profundas
por las que la técnica funciona produciendo buenos ajustes (no ponga el algoritmo); c)
Identifique sus principales debilidades; d) ¿Cuál es su capacidad de generalización comparado
con SVM?

\answer

\section*{Ejercicio 7}
\addtoc{Ejercicio 7}

\noindent Discuta pros y contras de los clasificadores SVM y Random Forest (RF). Considera que
SVM por su construcción a través de un problema de optimización debería ser un mejor
clasificador que RF. Justificar las respuestas.

\answer

\section*{Ejercicio 8}
\addtoc{Ejercicio 8}

\noindent ¿Cuál es a su criterio lo que permite a clasificadores como Random Forest basados en
un conjunto de clasificadores simples aprender de forma más eficiente? ¿Cuales son las
mejoras que introduce frente a los clasificadores simples? ¿Es Random Forest óptimo en
algún sentido? Justifique con precisión las contestaciones.

\answer

\section*{Ejercicio 9}
\addtoc{Ejercicio 9}

\noindent En un experimento para determinar la distribución del tamaño de los peces en un lago, se
decide echar una red para capturar una muestra representativa. Así se hace y se obtiene
una muestra suficientemente grande de la que se pueden obtener conclusiones estadísticas
sobre los peces del lago. Se obtiene la distribución de peces por tamaño y se entregan las
conclusiones. Discuta si las conclusiones obtenidas servirán para el objetivo que se persigue
e identifique si hay algo que lo impida.

\answer

\section*{Ejercicio 10}
\addtoc{Ejercicio 10}

\noindent Identifique que pasos daría y en que orden para conseguir con el menor esfuerzo posible un
buen modelo de red neuronal a partir una muestra de datos. Justifique los pasos propuestos,
el orden de los mismos y argumente que son adecuados para conseguir un buen óptimo.
Considere que tiene suficientes datos tanto para el ajuste como para el test.

\answer

\newpage

\begin{thebibliography}{5}

\bibitem{nombre-referencia}
Texto referencia
\\\url{https://url.referencia.com}

\end{thebibliography}

\end{document}

