\documentclass[11pt,a4paper]{article}
\usepackage[spanish]{babel}					% Utilizar español
\usepackage[utf8]{inputenc}					% Caracteres UTF-8
\usepackage{graphicx}						% Imagenes
\usepackage[hidelinks]{hyperref}			% Poner enlaces sin marcarlos en rojo
\usepackage{fancyhdr}						% Modificar encabezados y pies de pagina
\usepackage{float}							% Insertar figuras
\usepackage[textwidth=390pt]{geometry}		% Anchura de la pagina
\usepackage[nottoc]{tocbibind}				% Referencias (no incluir num pagina indice en Indice)
\usepackage{enumitem}
\usepackage[T1]{fontenc}

% Configuracion de encabezados y pies de pagina
\pagestyle{fancy}
\lhead{Vladislav Nikolov Vasilev}
\rhead{Asignatura}
\lfoot{Grado en Ingeniería Informática}
\cfoot{}
\rfoot{\thepage}
\renewcommand{\headrulewidth}{0.4pt}		% Linea cabeza de pagina
\renewcommand{\footrulewidth}{0.4pt}		% Linea pie de pagina

\begin{document}
\pagenumbering{gobble}

% Pagina de titulo
\begin{titlepage}

\begin{minipage}{\textwidth}

\centering

\includegraphics[scale=0.5]{img/ugr.png}\\

\textsc{\Large Aprendizaje Automático\\[0.2cm]}
\textsc{GRADO EN INGENIERÍA INFORMÁTICA}\\[1cm]

\noindent\rule[-1ex]{\textwidth}{1pt}\\[1.5ex]
\textsc{{\Huge PRÁCTICA 2\\[0.5ex]}}
\textsc{{\Large Programación\\}}
\noindent\rule[-1ex]{\textwidth}{2pt}\\[3.5ex]

\end{minipage}

\vspace{0.5cm}

\begin{minipage}{\textwidth}

\centering

\textbf{Autor}\\ {Vladislav Nikolov Vasilev}\\[2.5ex]
\textbf{Rama}\\ {Computación y Sistemas Inteligentes}\\[2.5ex]
\vspace{0.3cm}

\includegraphics[scale=0.3]{img/etsiit.jpeg}

\vspace{0.7cm}
\textsc{Escuela Técnica Superior de Ingenierías Informática y de Telecomunicación}\\
\vspace{1cm}
\textsc{Curso 2018-2019}
\end{minipage}
\end{titlepage}

\pagenumbering{arabic}
\tableofcontents
\thispagestyle{empty}				% No usar estilo en la pagina de indice

\newpage

\setlength{\parskip}{1em}

\section{\textsc{Ejercicio sobre la complejidad de H y el ruido}}
En este ejercicio debemos aprender la dificultad que introduce la aparición de ruido en las
etiquetas a la hora de elegir la clase de funciones más adecuada. Haremos uso de tres funciones
ya programadas:

\begin{itemize}
	\item $simula\_unif(N, dim, rango)$, que calcula una lista de N vectores de dimensión $dim$.
	Cada vector contiene $dim$ números aleatorios uniformes en el intervalo $rango$.
	\item $simula\_gaus(N, dim, sigma)$, que calcula una lista de longitud N de vectores de
	dimensión $dim$, donde cada posición del vector contiene un número aleatorio extraido de una
	distribucción Gaussiana de media 0 y varianza dada, para cada dimension, por la posición
	del vector $sigma$.
	\item $simula\_recta(intervalo)$, que simula de forma aleatoria los parámetros, $v = (a, b)$
	de una recta, $y = ax + b$, que corta al cuadrado $[-50, 50] \times [-50, 50]$.
\end{itemize}

\subsection*{Apartado 1}
\addcontentsline{toc}{subsection}{Apartado 1}

Dibujar una gráfica con la nube de puntos de salida correspondiente.

\begin{enumerate}[label=\textit{\alph*})]
	\item Considere $N = 50$, $dim = 2$, $rango = [-50, +50]$ con $simula_unif(N, dim, rango)$.
\end{enumerate}

\begin{enumerate}[resume,label=\textit{\alph*})]
	\item Considere $N = 50$, $dim = 2$ y $sigma = [5, 7]$ con $simula_gaus(N, dim, sigma)$.
\end{enumerate}


\newpage

\begin{thebibliography}{5}

\bibitem{nombre-referencia}
Texto referencia
\\\url{https://url.referencia.com}

\end{thebibliography}

\end{document}

