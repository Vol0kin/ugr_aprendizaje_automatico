\documentclass[11pt,a4paper]{article}
\usepackage[spanish,es-nodecimaldot]{babel}	% Utilizar español
\usepackage[utf8]{inputenc}					% Caracteres UTF-8
\usepackage{graphicx}						% Imagenes
\usepackage[hidelinks]{hyperref}			% Poner enlaces sin marcarlos en rojo
\usepackage{fancyhdr}						% Modificar encabezados y pies de pagina
\usepackage{float}							% Insertar figuras
\usepackage[textwidth=390pt]{geometry}		% Anchura de la pagina
\usepackage[nottoc]{tocbibind}				% Referencias (no incluir num pagina indice en Indice)
\usepackage{enumitem}
\usepackage[T1]{fontenc}

% Configuracion de encabezados y pies de pagina
\pagestyle{fancy}
\lhead{Vladislav Nikolov Vasilev}
\rhead{Asignatura}
\lfoot{Grado en Ingeniería Informática}
\cfoot{}
\rfoot{\thepage}
\renewcommand{\headrulewidth}{0.4pt}		% Linea cabeza de pagina
\renewcommand{\footrulewidth}{0.4pt}		% Linea pie de pagina

\begin{document}
\pagenumbering{gobble}

% Pagina de titulo
\begin{titlepage}

\begin{minipage}{\textwidth}

\centering

\includegraphics[scale=0.5]{img/ugr.png}\\

\textsc{\Large Aprendizaje Automático\\[0.2cm]}
\textsc{GRADO EN INGENIERÍA INFORMÁTICA}\\[1cm]

\noindent\rule[-1ex]{\textwidth}{1pt}\\[1.5ex]
\textsc{{\Huge PRÁCTICA 2\\[0.5ex]}}
\textsc{{\Large Programación\\}}
\noindent\rule[-1ex]{\textwidth}{2pt}\\[3.5ex]

\end{minipage}

\vspace{0.5cm}

\begin{minipage}{\textwidth}

\centering

\textbf{Autor}\\ {Vladislav Nikolov Vasilev}\\[2.5ex]
\textbf{Rama}\\ {Computación y Sistemas Inteligentes}\\[2.5ex]
\vspace{0.3cm}

\includegraphics[scale=0.3]{img/etsiit.jpeg}

\vspace{0.7cm}
\textsc{Escuela Técnica Superior de Ingenierías Informática y de Telecomunicación}\\
\vspace{1cm}
\textsc{Curso 2018-2019}
\end{minipage}
\end{titlepage}

\pagenumbering{arabic}
\tableofcontents
\thispagestyle{empty}				% No usar estilo en la pagina de indice

\newpage

\setlength{\parskip}{1em}

\section{\textsc{Ejercicio sobre la complejidad de H y el ruido}}
\noindent En este ejercicio debemos aprender la dificultad que introduce la aparición de ruido en las
etiquetas a la hora de elegir la clase de funciones más adecuada. Haremos uso de tres funciones
ya programadas:

\begin{itemize}
	\item $simula\_unif(N, dim, rango)$, que calcula una lista de N vectores de dimensión $dim$.
	Cada vector contiene $dim$ números aleatorios uniformes en el intervalo $rango$.
	\item $simula\_gaus(N, dim, sigma)$, que calcula una lista de longitud N de vectores de
	dimensión $dim$, donde cada posición del vector contiene un número aleatorio extraido de una
	distribucción Gaussiana de media 0 y varianza dada, para cada dimension, por la posición
	del vector $sigma$.
	\item $simula\_recta(intervalo)$, que simula de forma aleatoria los parámetros, $v = (a, b)$
	de una recta, $y = ax + b$, que corta al cuadrado $[-50, 50] \times [-50, 50]$.
\end{itemize}

\subsection*{Apartado 1}
\addcontentsline{toc}{subsection}{Apartado 1}

\noindent Dibujar una gráfica con la nube de puntos de salida correspondiente.

\begin{enumerate}[label=\textit{\alph*})]
	\item Considere $N = 50$, $dim = 2$, $rango = [-50, +50]$ con $simula\_unif(N, dim, rango)$.
\end{enumerate}

Como el código de la función $simula\_unif$ se ha proporcionado, no se va a mostrar su funcionamiento
porque se supone que ya es conocido. Habiendo dicho esto, se procede a mostrar los datos generados:

\begin{figure}[H]
\centering
\includegraphics[scale=0.53]{img/simula_unif.png}
\caption{Datos generados por la función $simula\_unif(N, dim, rango)$ con $N = 50$, $dim = 2$,
$rango = [-50, +50]$}
\end{figure}

\begin{enumerate}[resume,label=\textit{\alph*})]
	\item Considere $N = 50$, $dim = 2$ y $sigma = [5, 7]$ con $simula\_gaus(N, dim, sigma)$.
\end{enumerate}

De nuevo, como en el caso anterior, como ya se ha proporcionado la función $simula\_gaus$, no se va
a mostrar su funcionamiento porque se supone que ya es conocido. Así que, con esto en mente, podemos
mostrar los siguientes puntos generados:

\begin{figure}[H]
\centering
\includegraphics[scale=0.6]{img/simula_gaus.png}
\caption{Datos generados por la función $simula\_gaus(N, dim, sigma)$ con $N = 50$, $dim = 2$ y
$sigma = [5, 7]$}
\end{figure}

\subsection*{Apartado 2}
\addcontentsline{toc}{subsection}{Apartado 2}

\noindent Con ayuda de la función $simula\_unif()$ generar una muestra de puntos 2D a los
que vamos añadir una etiqueta usando el signo de la función $f(x, y) = y - ax - b$, es decir
el signo de la distancia de cada punto a la recta simulada con $simula\_recta()$.

\begin{enumerate}[label=\textit{\alph*})]
	\item Dibujar una gráfica donde los puntos muestren el resultado de su etiqueta, junto
	con la recta usada para ello. (Observe que todos los puntos están bien clasificados
	respecto de la recta).
\end{enumerate}

Para realizar esto, se nos ha proporcionado la función $simula\_recta()$ desde el principio, con lo
cuál no la vamos a comentar, ya que se supone conocida. Para clasificar los puntos, vamos a utilizar
una función llamada $f(x, y, a, b)$, donde $x$ es la coordenada $X$ del punto, $y$ la coordenada $Y$,
$a$ la pendiente de la recta simulada anteriormente y $b$ el termino independiente. Esta función
también se ha proporcionado, con lo cuál no se mostrará su implementación porque, de nuevo, se supone
conocida.

Los puntos generados, junto con la recta, son los siguientes:

\begin{figure}[H]
\centering
\includegraphics[scale=0.6]{img/data_line_no_noise.png}
\caption{Gráfica de puntos generados uniformemente con la recta que separa las dos clases.}
\end{figure}

Adicionalmente, para ver que la clasificación es la correcta, se ha calculado el ratio de puntos
clasificados correcta e incorrectamente mediante una función, la cuál se puede ver a continuación:

\begin{figure}[H]
\centering
\includegraphics[scale=0.4]{img/error_rate.png}
\caption{Función para el cálculo del ratio de aciertos y errores.}
\end{figure}

Los resultados obtenidos, como se espera en este caso ya que no hay ruido en la muestra, son los
siguientes:

\begin{figure}[H]
\centering
\includegraphics[scale=0.4]{img/error_no_noise.png}
\caption{Ratios de acierto y error obtenidos de la muestra sin ruido.}
\end{figure}

\begin{enumerate}[resume,label=\textit{\alph*})]
	\item Modifique de forma aleatoria un 10\% etiquetas positivas y otro 10\% de negativas
	y guarde los puntos con sus nuevas etiquetas. Dibuje de nuevo la gráfica anterior.
	(Ahora hay puntos mal clasificados respecto de la recta)
\end{enumerate}

Para insertar ruido en la muestra, de forma proporcional a la cantidad de etiquetas de las distintas
clases que tenemos, nos hemos ayudado de la siguiente función:

\begin{figure}[H]
\centering
\includegraphics[scale=0.4]{img/insert_noise.png}
\caption{Función para insertar ruido en una muestra de datos.}
\end{figure}

Al aplicar esta función sobre nuestras etiquetas, obtenemos los siguientes datos:

\begin{figure}[H]
\centering
\includegraphics[scale=0.6]{img/data_line_noise.png}
\caption{Gráfica de puntos generados uniformemente con ruido junto con la recta que los separa.}
\end{figure}

Como se puede ver, como había un mayor número de datos con etiqueta $-1$, se han modificado más
elementos de esta clase, siendo este número de elementos modificados proporcional respecto al número
de elementos totales de la clase.

Como era de esperarse, al haber modificado las etiquetas, los ratios de aciertos y error se han
visto modificados, dejándolos de la siguiente forma:

\begin{figure}[H]
\centering
\includegraphics[scale=0.7]{img/rates_noise.png}
\caption{Ratios de acierto y error obtenidos de la muestra con ruido.}
\end{figure}

\subsection*{Apartado 3}
\addcontentsline{toc}{subsection}{Apartado 3}
\noindent Supongamos ahora que las siguientes funciones definen la frontera de clasificación
de los puntos de la muestra en lugar de una recta:

\begin{enumerate}
	\item \label{itm:f1} $f(x, y) = (x - 10)^2 + (y - 20)^2 - 400$
	\item \label{itm:f2} $f(x, y) = 0.5(x + 10)^2 + (y - 20)^2 - 400$
	\item \label{itm:f3} $f(x, y) = 0.5(x - 10)^2 + (y + 20)^2 - 400$
	\item \label{itm:f4} $y - 20x^2 - 5x + 3$
\end{enumerate}

\noindent Visualizar el etiquetado generado en 2\textit{b} junto con cada una de las gráficas de cada
una de las funciones. Comparar las formas de las regiones positivas y negativas de estas nuevas
funciones con las obtenidas en el caso de la recta ¿Son estas funciones más complejas
mejores clasificadores que la función lineal? ¿En qué ganan a la función lineal? Explicar el
razonamiento.

Para ayudarnos con la visualización de los datos, hemos utilizado una función ya proporcionada,
con lo cuál no vamos a discutir como funciona. En vez de eso, veamos cómo son las funciones 
anteriores una vez que se han implementado:

\begin{figure}[H]
\centering
\includegraphics[scale=0.4]{img/functions.png}
\caption{Funciones implementadas en Python.}
\end{figure}

Estas funciones están pensadas para que se modifiquen todos los datos a la vez y se devuelva la
función aplicada a éstos.

Para calcular estos ratios, nos hemos ayudado de la siguiente función, la cuál ha sido parametrizada
para que pueda recibir cualquiera de las funciones anteriores, predecir las etiquetas y devolver
los correspondientes ratios:

\begin{figure}[H]
\centering
\includegraphics[scale=0.4]{img/error_rate_func.png}
\caption{Función para calcular las tasas de aciertos y error para las nuevas funciones.}
\end{figure}

A continuación, se presentan las gráficas de las funciones junto con los valors de los ratios
de aciertos y error:

\begin{figure}[H]
\centering
\begin{minipage}{.5\textwidth}
	\centering
	\includegraphics[scale=0.35]{img/f1.png}
	\caption{Gráfica y ratios de \ref{itm:f1}.}
\end{minipage}%
\begin{minipage}{.5\textwidth}
	\centering
	\includegraphics[scale=0.35]{img/f2.png}
	\caption{Gráfica y ratios de \ref{itm:f2}.}
\end{minipage}
\end{figure}

\begin{figure}[H]
\centering
\begin{minipage}{.5\textwidth}
	\centering
	\includegraphics[scale=0.35]{img/f3.png}
	\caption{Gráfica y ratios de \ref{itm:f3}.}
\end{minipage}%
\begin{minipage}{.5\textwidth}
	\centering
	\includegraphics[scale=0.35]{img/f4.png}
	\caption{Gráfica y ratios de \ref{itm:f4}.}
\end{minipage}
\end{figure}

Como se puede ver, son funciones más complejas que las lineales. Algunas de ellas son figuras
geométricas, como circunferencias o elipses (correspondientes a las funciones \ref{itm:f1} y
\ref{itm:f2}). Estas funciones serían capaces de explicar datos muy complejos, datos con los que
las funciones lineales tendrían problemas, debido a que no son linealmente separables mediante
una recta. Por ejemplo, si una clase estuviese dentro de otra, se podría usar perfectamente una
función como \ref{itm:f1} o \ref{itm:f2}.

Sin embargo, en este caso no parecen ser demasiado buenas, ya que en todos los casos los ratios
de aciertos son menores que los de la función lineal (recordemos que su ratio de aciertos era
$0.88$). Esto se debe a que, aunque los datos presenten un cierto ruido, se pueden llegar a separar
mejor utilizando una recta en vez de algo mucho más complejo como una elipse o dividiendo el espacio
en regiones como en el caso de \ref{itm:f3}.

También hay que considerar que en este caso estamos cogiendo directamente funciones, sin entrenar
ningún tipo de modelo para intentar ajustarlas a los datos. A lo mejor haciendo esto se podrían
obtener algunos resultados un poco mejores en algunas de las funciones, pero como existe ruido en la
muestra, es muy difícil dar con la función adecuada y que tenga el mínimo error. Además, en caso de
encontrarla, por haber recurrido a una clase de funciones muy compleja por ejemplo, nos podríamos
encontrar con el caso de \textit{overfitting}, lo cuál es un problema gravísimo y nada deseable.

En resumen, las funciones no lineales tienen sus puntos fuertes debido a que pueden explicar ciertos
tipos de muestras mejor de lo que lo haría una función lineal, pero no siempre son la solución a
nuestros problemas. Pueden darse casos, como por ejemplo este, que la función lineal le gane a todas
las no lineales, debido a que sea más fácil dividir los datos mediante una recta. Hay que ser muy
consciente de qué tipo de función elegir en cada momento. Por ejemplo, un comportamiento no lineal en
algunas carcterísticas nos podría indicar que sería mejor intentar recurrir a alguna función no lineal
que a una lineal para resolver el problema.

\end{document}

